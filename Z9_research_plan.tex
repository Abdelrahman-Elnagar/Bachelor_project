\documentclass[11pt,a4paper]{article}
\usepackage[utf8]{inputenc}
\usepackage[T1]{fontenc}
\usepackage{geometry}
\usepackage{amsmath}
\usepackage{amsfonts}
\usepackage{amssymb}
\usepackage{graphicx}
\usepackage{booktabs}
\usepackage{array}
\usepackage{longtable}
\usepackage{enumitem}
\usepackage{xcolor}
\usepackage{fancyhdr}
\usepackage{titlesec}
\usepackage{listings}
\usepackage{float}
\usepackage{tcolorbox}
\usepackage{parskip}
\usepackage{setspace}

% Color definitions
\definecolor{primaryblue}{rgb}{0.16,0.50,0.73}
\definecolor{secondaryblue}{rgb}{0.20,0.60,0.86}
\definecolor{accentorange}{rgb}{0.90,0.49,0.13}
\definecolor{lightgray}{rgb}{0.96,0.96,0.96}
\definecolor{darkgray}{rgb}{0.33,0.33,0.33}
\definecolor{successgreen}{rgb}{0.15,0.68,0.38}
\definecolor{warningred}{rgb}{0.91,0.30,0.24}

% Page setup
\geometry{margin=0.8in}
\setlength{\headheight}{14pt}
\pagestyle{fancy}
\fancyhf{}
\fancyhead[L]{\color{primaryblue}\textbf{Research Plan - Detailed \& Comprehensive Milestones}}
\fancyhead[R]{\color{darkgray}\thepage}
\renewcommand{\headrulewidth}{0.8pt}

% Title formatting with colors and spacing
\titleformat{\section}
{\color{primaryblue}\Large\bfseries}
{\color{primaryblue}\thesection}{1em}{}

\titleformat{\subsection}
{\color{secondaryblue}\large\bfseries}
{\color{secondaryblue}\thesubsection}{1em}{}

\titleformat{\subsubsection}
{\color{accentorange}\normalsize\bfseries}
{\color{accentorange}\thesubsubsection}{1em}{}

% Custom environments
\newtcolorbox{objectivebox}{
    colback=lightgray,
    colframe=primaryblue,
    boxrule=1pt,
    arc=3pt,
    left=5pt,
    right=5pt,
    top=5pt,
    bottom=5pt,
    fonttitle=\bfseries,
    title=\textbf{Objective}
}

\newtcolorbox{deliverablebox}{
    colback=lightgray,
    colframe=successgreen,
    boxrule=1pt,
    arc=3pt,
    left=5pt,
    right=5pt,
    top=5pt,
    bottom=5pt,
    fonttitle=\bfseries,
    title=\textbf{Deliverables}
}

\newtcolorbox{checkbox}{
    colback=lightgray,
    colframe=warningred,
    boxrule=1pt,
    arc=3pt,
    left=5pt,
    right=5pt,
    top=5pt,
    bottom=5pt,
    fonttitle=\bfseries,
    title=\textbf{Checks}
}

\newtcolorbox{codebox}{
    colback=lightgray,
    colframe=darkgray,
    boxrule=1pt,
    arc=3pt,
    left=5pt,
    right=5pt,
    top=5pt,
    bottom=5pt,
    fonttitle=\bfseries,
    title=\textbf{Code Example}
}

% Enhanced code listing setup
\lstset{
    basicstyle=\ttfamily\small,
    breaklines=true,
    frame=single,
    numbers=left,
    numberstyle=\tiny\color{darkgray},
    showstringspaces=false,
    backgroundcolor=\color{lightgray},
    keywordstyle=\color{primaryblue}\bfseries,
    commentstyle=\color{darkgray}\itshape,
    stringstyle=\color{accentorange},
    numberstyle=\color{darkgray}
}

% Enhanced list formatting
\setlist[enumerate,1]{label=\textbf{\color{primaryblue}\arabic*.},leftmargin=1.5em}
\setlist[enumerate,2]{label=\textbf{\color{secondaryblue}\alph*)},leftmargin=1.5em}
\setlist[itemize,1]{label=\textcolor{primaryblue}{$\bullet$},leftmargin=1.5em}
\setlist[itemize,2]{label=\textcolor{secondaryblue}{$\circ$},leftmargin=1.5em}

% Spacing
\setlength{\parskip}{0.5em}
\onehalfspacing

% Load hyperref last
\usepackage{hyperref}

\title{\textbf{\color{primaryblue}Research Plan - Detailed \& Comprehensive Milestones}}
\author{}
\date{}

\begin{document}

\begin{titlepage}
\centering
\vspace*{2cm}

{\Huge\bfseries\color{primaryblue}Research Plan}\\[0.5cm]
{\Large\color{secondaryblue}Detailed \& Comprehensive Milestones}\\[2cm]

\begin{tcolorbox}[colback=lightgray,colframe=primaryblue,boxrule=2pt,arc=5pt,width=0.8\textwidth]
\centering
\large\textbf{Weather Stability Impact on Model Performance Analysis}\\[0.3cm]
\small A systematic approach to quantifying the relationship between weather variability and predictive model accuracy
\end{tcolorbox}

\vspace{2cm}

\vfill
{\large\color{darkgray}Comprehensive Research Methodology}\\[0.2cm]
{\small\color{darkgray}10 Milestones • Statistical Analysis • Reproducible Research}
\end{titlepage}

\newpage
\tableofcontents
\newpage

\section{Milestone 1 - Define Research Scope, Objectives \& Success Criteria}

\begin{objectivebox}
Unambiguously state what you will measure and how success looks.
\end{objectivebox}

\subsection{Subtasks}

\begin{enumerate}[label=\arabic*.]
    \item Write a short project statement (3-5 sentences) that includes: goal, outcome, and why weather matters for this model.
    
    \item Enumerate research questions (RQ) and hypotheses (H). Example:
    \begin{itemize}
        \item \textbf{RQ1:} Do error metrics differ between weather-stable and weather-unstable periods?
        \item \textbf{H1:} MAE is higher during unstable-weather periods.
    \end{itemize}
    
    \item Define primary \& secondary metrics:
    \begin{itemize}
        \item \textbf{Primary:} MAE (mean absolute error)
        \item \textbf{Secondary:} RMSE, Bias (mean error), Error Std, 50/75/95\% absolute error percentiles, R² (if appropriate)
        \item If probabilistic model: CRPS, calibration (PIT)
    \end{itemize}
    
    \item Write success criteria (what ``meaningful'' effect looks like): e.g., Cohen's d ≥ 0.3 for MAE difference, or a consistent positive Spearman ρ ≥ 0.25 between WSI and MAE.
    
    \item Create an analysis plan document that lists tests and models you will run (pre-specified) - this prevents p-hacking.
\end{enumerate}

\begin{deliverablebox}
\begin{itemize}
    \item \texttt{docs/project\_statement.md}
    \item \texttt{docs/analysis\_plan.md} (includes RQs, metrics, success criteria)
\end{itemize}
\end{deliverablebox}

\section{Milestone 2 - Data Acquisition \& Inventory}

\begin{objectivebox}
Collect and inventory all data sources necessary for the analysis.
\end{objectivebox}

\subsection{Subtasks}

\begin{enumerate}[label=\arabic*.]
    \item Identify and access:
    \begin{itemize}
        \item \textbf{Prediction data:} \texttt{predictions.csv} with columns timestamp, location, model\_version, prediction, ground\_truth.
        \item \textbf{Weather data:} \texttt{weather.csv} with timestamp, location, temp\_C, temp\_min, temp\_max, humidity\_pct, precip\_mm, wind\_speed\_mps, pressure\_hPa, cloudiness\_pct.
        \item \textbf{Metadata:} \texttt{metadata.csv} with location\_lat, location\_lon, timezone, sensor\_notes.
    \end{itemize}
    
    \item Verify:
    \begin{itemize}
        \item Coverage (start/end timestamps, missing windows).
        \item Granularity (hourly/daily) - must match model granularity or be aggregatable.
    \end{itemize}
    
    \item Record data provenance (APIs, queries, file paths) in \texttt{data/README.md}.
    
    \item If using APIs (e.g., NOAA, OpenWeather), save exact request parameters and API responses (or scripts used to fetch).
\end{enumerate}

\begin{checkbox}
\begin{itemize}
    \item Compute row counts, unique timestamps, duplicates.
    \item Check that timestamp formats are ISO 8601 and timezone-aware.
    \item Create sample plot of timestamps to detect missing blocks.
\end{itemize}
\end{checkbox}

\begin{deliverablebox}
\begin{itemize}
    \item Raw files in \texttt{data/raw/} (or links if too large)
    \item \texttt{data/inventory.csv} (columns: filename, source, start, end, rows, notes)
\end{itemize}
\end{deliverablebox}

\section{Milestone 3 - Data Cleaning \& Synchronization}

\begin{objectivebox}
Make datasets analysis-ready and aligned.
\end{objectivebox}

\subsection{Subtasks}

\begin{enumerate}[label=\arabic*.]
    \item \textbf{Standardize timestamps:}
    \begin{itemize}
        \item Convert all timestamps to UTC or a single timezone. Use ISO format.
    \end{itemize}
    
    \item \textbf{Handle duplicates:}
    \begin{itemize}
        \item Keep most recent record where duplicates exist, or average duplicate sensors if appropriate.
    \end{itemize}
    
    \item \textbf{Missing data strategy:}
    \begin{itemize}
        \item Short gaps (< 3 timepoints): interpolate (linear or time-based).
        \item Medium gaps (3-24 timepoints): use rolling median or seasonal interpolation.
        \item Long gaps: mark excluded=True for downstream analysis.
        \item Log every imputation in \texttt{data/cleaning\_log.csv}.
    \end{itemize}
    
    \item \textbf{Outlier handling:}
    \begin{itemize}
        \item Flag values beyond domain-realistic thresholds (temp < -60°C or > 60°C etc.) as NaN.
        \item Use rolling z-score (window=24 for hourly) to flag transient spikes > 4σ.
    \end{itemize}
    
    \item Merge predictions with weather on (timestamp, location) using an outer join and then drop timestamps where ground\_truth is missing (unless you plan to analyze predicted-only behavior).
    
    \item Create a harmonized dataset \texttt{data/processed/aligned.csv} with columns:
    \begin{verbatim}
    timestamp, location, model_version, prediction, ground_truth,
    error, abs_error, temp_mean, temp_min, temp_max, temp_std, humidity, 
    precip_mm, precip_flag, wind_speed, pressure, cloudiness, excluded_flag
    \end{verbatim}
\end{enumerate}

\begin{checkbox}
\begin{itemize}
    \item Report number/percentage of dropped rows and imputed rows.
    \item Plot missingness heatmap (timestamps × variables).
    \item Save \texttt{data/cleaning\_log.csv} with rows: action, affected\_rows, rationale.
\end{itemize}
\end{checkbox}

\begin{deliverablebox}
\begin{itemize}
    \item \texttt{data/processed/aligned.csv}
    \item \texttt{data/cleaning\_log.csv}
    \item small summary notebook \texttt{notebooks/01\_data\_cleaning.ipynb}
\end{itemize}
\end{deliverablebox}

\section{Milestone 4 - Weather Feature Engineering}

\begin{objectivebox}
Compute features that capture variability, trends, and extremes for WSI.
\end{objectivebox}

\subsection{Core features (per timestamp)}
\begin{itemize}
    \item temp\_mean, temp\_min, temp\_max
    \item temp\_std = rolling std over past 24h (or appropriate window)
    \item abs\_temp\_change = abs(temp\_mean - temp\_mean\_lag1)
    \item precip\_mm, precip\_flag = 1 if precip\_mm > 0
    \item precip\_rolling\_sum\_3d = total precip last 72 hours
    \item wind\_max = max wind in last 24h
    \item pressure\_change = pressure - pressure\_lag24
    \item humid\_mean, humid\_std = rolling stats
    \item extreme\_flag = logical OR of thresholds (e.g., temp\_std>t1 or precip\_flag or wind\_max>t2)
\end{itemize}

\subsection{Feature engineering steps}
\begin{enumerate}[label=\arabic*.]
    \item Choose windows (e.g., 24h, 72h, 7d) consistent with system dynamics.
    \item Robust scaling: compute median and IQR for each feature and create feature\_robust = (feature - median)/IQR.
    \item Save scaling parameters to \texttt{models/scalers.json}.
\end{enumerate}

\begin{deliverablebox}
\begin{itemize}
    \item \texttt{data/features/weather\_features.csv}
    \item \texttt{models/scalers.json}
    \item \texttt{notebooks/02\_feature\_engineering.ipynb} with exploratory plots (distributions and pairwise correlations)
\end{itemize}
\end{deliverablebox}

\section{Milestone 5 - Compute Weather Stability Index (WSI)}

\begin{objectivebox}
Produce a reproducible continuous WSI and a categorical stable/unstable label.
\end{objectivebox}

\subsection{WSI construction (recommended starting formula)}

\begin{enumerate}[label=\arabic*.]
    \item Select normalized features oriented so higher = more instability:
    \begin{itemize}
        \item temp\_std\_norm, abs\_temp\_change\_norm, precip\_3d\_norm, wind\_max\_norm, pressure\_change\_norm, humid\_std\_norm.
    \end{itemize}
    
    \item Define equal weights initially:
    \begin{equation}
    \text{WSI} = \frac{\text{temp\_std\_norm} + \text{abs\_temp\_change\_norm} + \text{precip\_3d\_norm} + \text{wind\_max\_norm} + \text{pressure\_change\_norm} + \text{humid\_std\_norm}}{6}
    \end{equation}
    
    \item Save WSI as continuous score, then compute WSI\_z = (WSI - mean)/std.
\end{enumerate}

\subsection{Classification to stable/unstable}
\begin{itemize}
    \item \textbf{Primary method:} KMeans with n\_clusters=2 on the normalized feature space; label cluster with higher mean WSI as unstable.
    \item \textbf{Secondary method:} Percentile threshold: unstable if WSI ≥ percentile(75) (sensitivity tests use 70/80).
    \item \textbf{Alternative:} Fit PCA and use first PC as WSI surrogate (document differences).
\end{itemize}

\begin{deliverablebox}
\begin{itemize}
    \item \texttt{data/processed/wsi.csv} with WSI, WSI\_z, and stability\_label\_kmeans, stability\_label\_percentile.
    \item \texttt{notebooks/03\_wsi.ipynb} describing choice, showing cluster centroids and feature contributions.
\end{itemize}
\end{deliverablebox}

\begin{checkbox}
\begin{itemize}
    \item Inspect cluster sizes (not too imbalanced).
    \item Show examples of days labeled unstable and inspect weather attributes manually to verify plausibility.
\end{itemize}
\end{checkbox}

\section{Milestone 6 - Model Performance Metric Computation}

\begin{objectivebox}
Compute daily/hourly performance timelines aligned with WSI.
\end{objectivebox}

\subsection{Core metrics (per timestamp or aggregated per day)}
\begin{itemize}
    \item error = prediction - ground\_truth
    \item abs\_error
    \item MAE - average of abs\_error across unit (day/hour)
    \item RMSE - sqrt of mean squared error
    \item Bias - mean of error
    \item ErrorStd - std of error
    \item Percentile absolute errors: 50th, 75th, 95th
    \item rolling\_MAE\_7d - smooth to reveal trends
\end{itemize}

\subsection{Design notes}
\begin{itemize}
    \item Compute both instantaneous metrics and rolling-window metrics (7-day, 14-day).
    \item If model versions change mid-year, compute metrics per model\_version separately and include model\_version as covariate.
\end{itemize}

\begin{deliverablebox}
\begin{itemize}
    \item \texttt{data/metrics/performance\_metrics.csv} (timestamp, location, MAE, RMSE, Bias, ErrorStd, ...).
    \item \texttt{notebooks/04\_metrics.ipynb} with plots of raw \& smoothed metrics.
\end{itemize}
\end{deliverablebox}

\section{Milestone 7 - Comparative Analysis \& Statistical Testing}

\begin{objectivebox}
Test associations and differences between weather stability and model performance.
\end{objectivebox}

\subsection{Analyses to run (pre-specified)}

\subsubsection{Descriptive stats}
\begin{itemize}
    \item Grouped summary: mean/median/CI of MAE for stable vs unstable.
    \item Effect size: Cohen's d (for means) and Cliff's delta (nonparametric).
\end{itemize}

\subsubsection{Hypothesis testing}
\begin{itemize}
    \item Check normality (Shapiro) and equal variance (Levene).
    \item If normal-ish and variances equal: two-sample t-test (Welch if unequal variances).
    \item Else: Mann-Whitney U test.
\end{itemize}

\subsubsection{Time-series aware testing}
\begin{itemize}
    \item Block bootstrap (block length = 7 days) to compute CI for mean difference to account for autocorrelation.
    \item Autocorrelation check: plot ACF of errors and compute effective sample size.
\end{itemize}

\subsubsection{Regression analysis}
\begin{itemize}
    \item OLS (MAE ∼ WSI + day\_of\_week + month + model\_version), use HAC robust SEs (cov\_type='HAC', lags=7).
    \item Mixed-effects model if multiple locations: MAE ∼ WSI + (1|location) (random intercepts).
    \item Nonlinear model: GAM (MAE ∼ s(WSI) + s(temp\_std) + ...) or RandomForest with SHAP explanations.
\end{itemize}

\subsubsection{Lagged analysis}
\begin{itemize}
    \item Cross-correlation function (CCF) between WSI and MAE for lags -L to +L (e.g., L=30 days).
    \item Granger causality tests (careful with assumptions) to test predictive direction.
\end{itemize}

\subsubsection{Change-point detection}
\begin{itemize}
    \item Use ruptures to find changes in MAE mean/variance and compare change dates to peaks in WSI.
\end{itemize}

\begin{deliverablebox}
\begin{itemize}
    \item \texttt{results/stat\_tests.csv} documenting test names, statistics, p-values, conclusions.
    \item \texttt{notebooks/05\_analysis.ipynb} with full code of analyses.
\end{itemize}
\end{deliverablebox}

\section{Milestone 8 - Visualization, Figures \& Interpretation}

\begin{objectivebox}
Produce a compact set of publication-quality figures and an interpretive narrative.
\end{objectivebox}

\subsection{Core figures}

\subsubsection{Dual timeline plot}
\begin{itemize}
    \item Upper panel: continuous WSI (line) with shading for unstable segments.
    \item Lower panel: rolling\_MAE\_7d (line) with vertical markers for change-points.
    \item Shared x-axis: timestamp (year).
\end{itemize}

\subsubsection{Boxplot / Violin}
\begin{itemize}
    \item MAE distributions for stable vs unstable with means and 95\% CIs.
\end{itemize}

\subsubsection{Scatter + smoothing}
\begin{itemize}
    \item WSI vs MAE scatter with LOESS smoothing line; annotate Spearman ρ and p-value.
\end{itemize}

\subsubsection{Lag correlation}
\begin{itemize}
    \item Bar chart of cross-correlation coefficients at lags -L..L.
\end{itemize}

\subsubsection{Calendar heatmap}
\begin{itemize}
    \item Calendar-style heatmap of daily MAE with overlay color indicating stability\_label.
\end{itemize}

\subsubsection{Model explanation}
\begin{itemize}
    \item Feature importance / SHAP summary from RandomForest or XGBoost fitted to explain abs\_error.
\end{itemize}

\subsection{Design \& export}
\begin{itemize}
    \item Use consistent fonts, labels, and colorblind-friendly palettes.
    \item Include figure captions that state the dataset used, smoothing parameters, and sample sizes.
    \item Export high-resolution PNGs and a vector PDF for each figure in \texttt{figures/}.
\end{itemize}

\begin{deliverablebox}
\begin{itemize}
    \item \texttt{figures/*} (PNG + PDF)
    \item \texttt{notebooks/06\_visualization.ipynb} with plotting code and figure exports
\end{itemize}
\end{deliverablebox}

\section{Milestone 9 - Robustness \& Sensitivity Analyses}

\begin{objectivebox}
Show that findings aren't artifacts of a particular WSI definition or analysis choice.
\end{objectivebox}

\subsection{Robustness checks}

Recompute results using:
\begin{itemize}
    \item Different WSI weighting schemes (equal weights, PCA-based weights, regression-weighted).
    \item Different classification thresholds (70th, 75th, 80th percentile).
    \item Alternate clustering (GaussianMixture vs KMeans).
    \item Seasonal sub-analyses: restrict to winter vs summer.
    \item Excluding days with missing or imputed ground\_truth.
    \item Re-run key tests using block bootstrap; report bootstrap CIs.
    \item If model updated mid-year: run analyses separately for each model version.
\end{itemize}

\begin{deliverablebox}
\begin{itemize}
    \item \texttt{results/robustness\_summary.md} - table of sensitivity experiments and outcomes.
    \item \texttt{notebooks/07\_robustness.ipynb}
\end{itemize}
\end{deliverablebox}

\section{Milestone 10 - Reporting, Reproducibility \& Handover}

\begin{objectivebox}
Produce reproducible artifacts and a final report suitable for submission.
\end{objectivebox}

\subsection{Report components}
Abstract (short), Introduction (motivation), Methods (data + WSI computation), Results (statistics \& figures), Discussion (interpretation, limitations), Conclusion, Appendix (robustness), Code availability.

Save as \texttt{report/final\_report.pdf} and \texttt{report/final\_report.docx} (if needed).

\subsection{Reproducibility checklist}
\begin{itemize}
    \item \texttt{requirements.txt} (pinned versions)
    \item \texttt{environment.yml} (optional conda)
    \item \texttt{run\_all.sh} - script to run all notebooks end-to-end and re-generate figures.
    \item \texttt{README.md} with instructions: how to reproduce, expected runtimes, hardware needs.
    \item Save seeds and parameters: \texttt{config/params.json} (WSI formula, thresholds, random seeds).
\end{itemize}

Use Git commits with messages mapping to milestones. Example commit messages:
\begin{itemize}
    \item \texttt{feat(data): add raw predictions and weather files}
    \item \texttt{fix(cleaning): impute small gaps; log actions}
    \item \texttt{feat(wsi): implement WSI and kmeans labeling}
    \item \texttt{chore(report): add draft of methods section}
\end{itemize}

\begin{deliverablebox}
\begin{itemize}
    \item \texttt{report/final\_report.pdf}
    \item \texttt{notebooks/} (all notebooks)
    \item \texttt{requirements.txt}, \texttt{run\_all.sh}, \texttt{config/params.json}
    \item \texttt{archive/} (zipped package if delivering)
\end{itemize}
\end{deliverablebox}

\section{Additional Practical Guidance \& Code Snippets}

\begin{codebox}
\textbf{Pandas merging \& error calculation}
\begin{lstlisting}[language=Python]
import pandas as pd
pred = pd.read_csv('data/raw/predictions.csv', parse_dates=['timestamp'])
weather = pd.read_csv('data/raw/weather.csv', parse_dates=['timestamp'])
df = pred.merge(weather, on=['timestamp','location'], how='outer')
df['error'] = df['prediction'] - df['ground_truth']
df['abs_error'] = df['error'].abs()
df.to_csv('data/processed/aligned.csv', index=False)
\end{lstlisting}
\end{codebox}

\begin{codebox}
\textbf{WSI (robust scaling + sum)}
\begin{lstlisting}[language=Python]
from sklearn.preprocessing import RobustScaler
cols = ['temp_std','abs_temp_change','precip_3d','wind_max','pressure_change','humid_std']
scaler = RobustScaler()
X = scaler.fit_transform(df[cols].fillna(0))
df['WSI'] = X.sum(axis=1) / X.shape[1]
# save scaler params
import joblib
joblib.dump(scaler, 'models/scaler_wsi.pkl')
\end{lstlisting}
\end{codebox}

\begin{codebox}
\textbf{KMeans labeling}
\begin{lstlisting}[language=Python]
from sklearn.cluster import KMeans
kmeans = KMeans(n_clusters=2, random_state=0).fit(X)
df['wsi_cluster'] = kmeans.labels_
# map label to unstable (cluster with higher mean WSI)
cluster_means = df.groupby('wsi_cluster')['WSI'].mean()
unstable_label = cluster_means.idxmax()
df['stability_label_kmeans'] = df['wsi_cluster'] == unstable_label
\end{lstlisting}
\end{codebox}

\begin{codebox}
\textbf{Block bootstrap (example skeleton)}
\begin{lstlisting}[language=Python]
import numpy as np
def block_bootstrap(series, block_size, n_boot):
    n = len(series)
    indices = np.arange(n)
    boot_stats = []
    for _ in range(n_boot):
        starts = np.random.randint(0, n-block_size+1, size=int(np.ceil(n/block_size)))
        idx = np.concatenate([np.arange(s, s+block_size) for s in starts])[:n]
        boot_series = series.iloc[idx].reset_index(drop=True)
        boot_stats.append(boot_series.mean())
    return np.array(boot_stats)
\end{lstlisting}
\end{codebox}

\section{Risks, Mitigations \& Notes}

\begin{itemize}
    \item \textbf{Model version drift mid-year:} include model\_version in analyses or split the year.
    \item \textbf{Autocorrelation \& seasonality:} use block bootstrap, include seasonal covariates, or fit time-series models.
    \item \textbf{Sparse extreme events:} use event-study averaging instead of per-day stats when events are rare.
    \item \textbf{Sensor/ground-truth outages:} transparently exclude or flag these periods and report how many days omitted.
\end{itemize}

\section{Suggested Folder Structure (final)}

\begin{verbatim}
project/
├─ data/
│  ├─ raw/
│  ├─ processed/
├─ notebooks/
│  ├─ 01_data_cleaning.ipynb
│  ├─ 02_feature_engineering.ipynb
│  ├─ ...
├─ models/
├─ figures/
├─ results/
├─ report/
├─ config/
│  └─ params.json
├─ requirements.txt
└─ run_all.sh
\end{verbatim}

\end{document}
