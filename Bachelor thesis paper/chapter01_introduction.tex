\chapter{Introduction}
\label{chap:introduction}

\section{Motivation}
\label{sec:motivation}

Renewable energy sources, particularly solar photovoltaics (PV) and wind power, have become integral components of Germany's energy transition towards a carbon-neutral future. As the share of renewable energy in the electricity grid continues to grow, accurate forecasting of renewable energy production has become critical for grid stability, energy market operations, and optimal resource allocation~\cite{giebel2011state, antonanzas2016review}. However, the inherent variability and uncertainty associated with weather-dependent renewable energy generation pose significant challenges for prediction models.

Weather conditions exhibit substantial temporal variability, transitioning between relatively stable periods characterized by consistent meteorological patterns and unstable periods marked by rapid changes in atmospheric conditions. These weather regimes directly influence renewable energy production: stable weather typically results in more predictable generation patterns, while unstable weather introduces greater uncertainty and variability. Yet, the extent to which weather stability affects the performance of forecasting models (both statistical and deep learning) remains insufficiently explored in the existing literature.

Grid operators and energy market participants require reliable forecasts under all weather conditions to ensure system stability, optimize dispatch decisions, and manage grid congestion effectively~\cite{antonanzas2016review, zhang2019short}. Current operational forecasting systems often employ a single model or ensemble approach without considering the prevailing weather regime~\cite{wang2021comparative}. This one-size-fits-all strategy may be suboptimal, as different prediction models may exhibit varying levels of robustness to weather instability. A model that performs excellently under stable conditions may degrade significantly during unstable periods, potentially compromising operational reliability.

Despite the recognized importance of weather variability in renewable energy forecasting, there is a notable gap in the literature regarding systematic evaluation of model robustness across different weather stability regimes. Most studies evaluate model performance on aggregate datasets without explicitly accounting for weather regime characteristics~\cite{ahmed2020review, pombo2022benchmarking}. This research addresses this gap by developing a comprehensive framework to quantify weather stability and assess how both statistical and deep learning prediction models respond to stable versus unstable weather conditions. The framework builds upon established weather regime classification methods~\cite{huth2008classifications, michelangeli1995weather, vautard1990multiple} and applies them to renewable energy forecasting contexts.

The significance of this research extends beyond academic interest. By identifying which models are most robust to weather instability and providing operational guidance on model selection based on weather conditions, this work contributes to improving the reliability and efficiency of renewable energy integration. The findings can inform grid operators, energy traders, and forecasting system designers about optimal model selection strategies that adapt to changing weather conditions, ultimately supporting more reliable and cost-effective renewable energy management.

\section{Research Questions}
\label{sec:research_questions}

To address the research gap identified in Section~\ref{sec:motivation}, this thesis investigates the following research questions:

\begin{enumerate}
    \item \textbf{RQ1:} Does renewable energy prediction accuracy differ significantly between weather-stable and weather-unstable periods?
    
    This question seeks to establish whether weather stability is a meaningful factor influencing prediction model performance. A positive answer would confirm that weather regime characteristics affect forecasting accuracy, justifying the development of weather-adaptive model selection strategies.
    
    \item \textbf{RQ2:} Which forecasting models (statistical and deep learning) are most robust to weather instability (i.e., show smallest performance degradation during unstable periods)? 
    
    This question aims to identify which models maintain their predictive performance when weather conditions become unstable. Understanding model-specific robustness characteristics enables informed selection of appropriate forecasting models based on expected weather conditions.
    
    \item \textbf{RQ3:} Can weather stability information improve model selection strategies for operational renewable energy forecasting?
    
    This question evaluates the practical utility of incorporating weather stability information into operational forecasting systems. It assesses whether knowledge of weather regime characteristics can be leveraged to select optimal models dynamically, potentially improving overall forecasting accuracy and reliability.
\end{enumerate}

These research questions guide the development of the dual-pipeline methodology described in Chapter~\ref{chap:methodology}, where weather stability classification and energy prediction models are evaluated independently before being combined for comparative analysis.

\section{Objectives}
\label{sec:objectives}

To address the research questions outlined in Section~\ref{sec:research_questions}, this research aims to achieve the following objectives:

\begin{enumerate}
    \item \textbf{Develop a Weather Stability Index (WSI)} using 11 weather attributes to classify periods as stable or unstable. This index will be computed from meteorological data including temperature, cloudiness, wind, precipitation, pressure, and other relevant atmospheric variables. The WSI will provide a quantitative measure of weather variability, enabling systematic classification of weather regimes throughout the analysis period.
    
    \item \textbf{Implement and evaluate statistical and deep learning models} to predict renewable energy production for 2024. Multiple established forecasting models will be implemented, including statistical models (persistence models, ARIMA/SARIMA, Prophet, exponential smoothing) and deep learning models (LSTM, CNN-LSTM, TCN, and ensemble approaches). All models will be trained and evaluated using historical renewable energy production data for Germany.
    
    \item \textbf{Compare model performance under stable vs unstable weather conditions} by stratifying prediction errors according to the weather stability classification. This comparison will quantify performance differences and identify which models are most affected by weather variability.
    
    \item \textbf{Identify which models are most robust to weather instability} by analyzing performance degradation patterns. Models that maintain relatively stable accuracy across different weather regimes will be considered more robust and potentially more suitable for operational deployment.
    
    \item \textbf{Provide operational recommendations on model selection} based on weather conditions. The findings will be synthesized into actionable guidance for grid operators and forecasting system designers, suggesting optimal model selection strategies that adapt to prevailing weather stability characteristics.
\end{enumerate}

These objectives collectively support the overarching goal of improving renewable energy forecasting reliability through weather-aware model selection, contributing to more effective renewable energy integration and grid management.

\section{Thesis Structure}
\label{sec:thesis_structure}

This thesis is organized into seven main chapters, following a logical progression from problem identification through methodology development to analysis and conclusions.

\textbf{Chapter~\ref{chap:introduction} (Introduction)} establishes the research context, motivation, and objectives. This chapter presents the research questions and outlines the overall structure of the thesis.

\textbf{Chapter~\ref{chap:literature} (Literature Review)} reviews existing research on renewable energy forecasting, weather regime classification, and model evaluation methodologies. It identifies gaps in current knowledge and situates this research within the broader academic discourse.

\textbf{Chapter~\ref{chap:methodology} (Methodology)} presents the research design and methodological framework. The chapter describes the dual-pipeline architecture, which consists of (1) a Weather Stability Classification pipeline that processes meteorological data to identify stable and unstable periods, (2) a Renewable Energy Prediction Models pipeline that implements and evaluates both statistical and deep learning forecasting models, and (3) a Comparative Analysis pipeline that merges stability classifications with model performance metrics to assess robustness. This chapter also details data collection procedures, preprocessing steps, and analysis methods.

\textbf{Chapter~\ref{chap:data} (Data and Experimental Setup)} provides comprehensive documentation of the datasets used in this research, including weather data from the German Meteorological Service (DWD) and renewable energy production data. The chapter describes data sources, collection procedures, quality assurance measures, and experimental configuration.

\textbf{Chapter~\ref{chap:results} (Results)} presents the empirical findings, including Weather Stability Index characteristics, model performance metrics stratified by weather regime, statistical test results, and model robustness rankings. Results are presented through descriptive statistics, tables, and visualizations.

\textbf{Chapter~\ref{chap:discussion} (Discussion)} interprets the results in the context of existing literature, discusses implications for operational forecasting systems, acknowledges limitations of the study, and explores the practical significance of the findings.

\textbf{Chapter~\ref{chap:conclusion} (Conclusion)} summarizes the main contributions of this research, revisits the research questions, discusses their resolution, and suggests directions for future work.

The thesis follows a structured approach where the dual-pipeline methodology enables systematic evaluation of model robustness to weather instability, ultimately converging on comparative analysis that addresses all three research questions.