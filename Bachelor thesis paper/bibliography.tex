\begin{thebibliography}{99}

\bibitem{giebel2011state}
G. Giebel, R. Brownsword, G. Kariniotakis, M. Denhard, and C. Draxl, ``The state of the art in short-term prediction of wind power: A literature overview,'' ANEMOS.plus, Tech. Rep., 2011, pp. 1--100.

\bibitem{antonanzas2016review}
J. Antonanzas, N. Osorio, R. Escobar, R. Urraca, F. J. Mart{\'i}nez-de-Pis{\'o}n, and A. Sanz-Garc{\'i}a, ``Review of photovoltaic power forecasting,'' \textit{Solar Energy}, vol. 136, pp. 78--111, 2016.

\bibitem{zhang2019short}
Y. Zhang, J. Wang, and X. Wang, ``Short-term wind speed prediction based on spatial correlation and artificial neural networks,'' \textit{Journal of Wind Engineering and Industrial Aerodynamics}, vol. 186, pp. 17--25, 2019.

\bibitem{holton2004introduction}
J. R. Holton, \textit{An Introduction to Dynamic Meteorology}, 4th ed. Elsevier Academic Press, 2004.


\bibitem{huth2008classifications}
R. Huth, C. Beck, A. Philipp, M. Demuzere, Z. Ustrnul, M. Cahynov{\'a}, J. Kysel{\'y}, and O. E. Tveito, ``Classifications of atmospheric circulation patterns: recent advances and applications,'' \textit{Annals of the New York Academy of Sciences}, vol. 1146, no. 1, pp. 105--152, 2008.

\bibitem{michelangeli1995weather}
P.-A. Michelangeli, R. Vautard, and B. Legras, ``Weather regimes: Recurrence and quasi stationarity,'' \textit{Journal of the Atmospheric Sciences}, vol. 52, no. 8, pp. 1237--1256, 1995.

\bibitem{vautard1990multiple}
R. Vautard, ``Multiple weather regimes over the North Atlantic: Analysis of precursors and successors,'' \textit{Monthly Weather Review}, vol. 118, no. 10, pp. 2056--2081, 1990.

\bibitem{mclachlan2000finite}
G. J. McLachlan and D. Peel, \textit{Finite mixture models}. New York, NY, USA: John Wiley \& Sons, 2000.

\bibitem{rabiner1989tutorial}
L. R. Rabiner, ``A tutorial on hidden Markov models and selected applications in speech recognition,'' \textit{Proceedings of the IEEE}, vol. 77, no. 2, pp. 257--286, 1989.

\bibitem{mann1947test}
H. B. Mann and D. R. Whitney, ``On a test of whether one of two random variables is stochastically larger than the other,'' \textit{The Annals of Mathematical Statistics}, vol. 18, no. 1, pp. 50--60, 1947.

\bibitem{cohen1988statistical}
J. Cohen, \textit{Statistical power analysis for the behavioral sciences}, 2nd ed. Hillsdale, NJ, USA: Routledge, 1988.

\bibitem{benjamini1995controlling}
Y. Benjamini and Y. Hochberg, ``Controlling the false discovery rate: a practical and powerful approach to multiple testing,'' \textit{Journal of the Royal Statistical Society: Series B (Methodological)}, vol. 57, no. 1, pp. 289--300, 1995.

\bibitem{huber1981robust}
P. J. Huber, \textit{Robust Statistics}. John Wiley & Sons, 1981.

\bibitem{taylor2018forecasting}
S. J. Taylor and B. Letham, ``Forecasting at scale,'' \textit{The American Statistician}, vol. 72, no. 1, pp. 37--45, 2018.

\bibitem{sedai2023performance}
A. Sedai, R. Dhakal, S. Gautam, A. Dhamala, A. Bilbao, Q. Wang, A. Wigington, and S. Pol, ``Performance Analysis of Statistical, Machine Learning and Deep Learning Models in Long-Term Forecasting of Solar Power Production,'' \textit{Forecasting}, vol. 5, no. 1, 2023.

\bibitem{cabello2023forecasting}
T. Cabello-L{\'o}pez, M. Carranza-Garc{\'i}a, J. Riquelme, and J. Garc{\'i}a-Gutierrez, ``Forecasting solar energy production in Spain: A comparison of univariate and multivariate models at the national level,'' \textit{Applied Energy}, vol. 341, p. 121645, 2023.

\bibitem{alkandari2020solar}
M. AlKandari and I. Ahmad, ``Solar power generation forecasting using ensemble approach based on deep learning and statistical methods,'' \textit{Applied Computing and Informatics}, 2020.

\bibitem{devaraj2021holistic}
J. Devaraj, R. Elavarasan, G. Shafiullah, T. Jamal, and I. Khan, ``A holistic review on energy forecasting using big data and deep learning models,'' \textit{International Journal of Energy Research}, vol. 45, no. 9, pp. 13489--13530, 2021.

\bibitem{alkhayat2021review}
G. Alkhayat and R. Mehmood, ``A review and taxonomy of wind and solar energy forecasting methods based on deep learning,'' \textit{Energy and AI}, vol. 5, p. 100060, 2021.

\bibitem{wang2019review}
H. Wang, Z. Lei, X. Zhang, B. Zhou, and J. Peng, ``A review of deep learning for renewable energy forecasting,'' \textit{Energy Conversion and Management}, vol. 198, p. 111799, 2019.

\bibitem{haider2022deep}
S. Haider, M. Sajid, H. Sajid, E. Uddin, and Y. Ayaz, ``Deep learning and statistical methods for short- and long-term solar irradiance forecasting for Islamabad,'' \textit{Renewable Energy}, 2022.

\bibitem{luo2021deep}
X. Luo, D. Zhang, and X. Zhu, ``Deep learning based forecasting of photovoltaic power generation by incorporating domain knowledge,'' \textit{Energy}, vol. 225, p. 120240, 2021.

\bibitem{husein2024towards}
M. Husein, E. Gago, B. Hasan, and M. Pegalajar, ``Towards energy efficiency: A comprehensive review of deep learning-based photovoltaic power forecasting strategies,'' \textit{Heliyon}, vol. 10, 2024.

\bibitem{mirza2023quantile}
A. Mirza, Z. Shu, M. Usman, M. Mansoor, and Q. Ling, ``Quantile-transformed multi-attention residual framework (QT-MARF) for medium-term PV and wind power prediction,'' \textit{Renewable Energy}, 2023.

\bibitem{li2020hybrid}
P. Li, K. Zhou, X. Lu, and S. Yang, ``A hybrid deep learning model for short-term PV power forecasting,'' \textit{Applied Energy}, vol. 259, p. 114216, 2020.

\bibitem{jamil2023predictive}
I. Jamil, H. Lucheng, S. Iqbal, M. Aurangzalb, R. Jamil, H. Kotb, A. Alkuhayli, and K. AboRas, ``Predictive evaluation of solar energy variables for a large-scale solar power plant based on triple deep learning forecast models,'' \textit{Alexandria Engineering Journal}, 2023.

\bibitem{venkateswaran2024efficient}
D. Venkateswaran and Y. Cho, ``Efficient solar power generation forecasting for greenhouses: A hybrid deep learning approach,'' \textit{Alexandria Engineering Journal}, 2024.

\bibitem{wang2021comparative}
X. Wang, Y. Sun, D. Luo, and J. Peng, ``Comparative study of machine learning approaches for predicting short-term photovoltaic power output based on weather type classification,'' \textit{Energy}, vol. 231, p. 122733, 2021.

\bibitem{cervantes2025heuristic}
I. Cervantes, C. Cervantes-Ortiz, D. V{\'a}zquez-Santana, and A. Arguelles, ``Heuristic-machine learning models for solar radiation forecasting in K{\"o}ppen climate zones,'' \textit{Applied Soft Computing}, vol. 171, p. 112807, 2025.

\bibitem{serras2024optimizing}
F. Serras, K. Vandelanotte, R. Borgers, B. Van Schaeybroeck, P. Termonia, M. Demuzere, and N. Van Lipzig, ``Optimizing climate model selection in regional studies using an adaptive weather type based framework: a case study for extreme heat in Belgium,'' \textit{Climate Dynamics}, 2024.

\bibitem{jain2022novel}
S. Jain, ``A novel seasonal segmentation approach for day-ahead load forecasting,'' \textit{Energy}, vol. 258, p. 124752, 2022.

\bibitem{blazakis2024towards}
K. Blazakis, N. Schetakis, P. Bonfini, K. Stavrakakis, E. Karapidakis, and Y. Katsigiannis, ``Towards Automated Model Selection for Wind Speed and Solar Irradiance Forecasting,'' \textit{Sensors}, vol. 24, no. 15, p. 5035, 2024.

\bibitem{lim2022solar}
S. Lim, J. Huh, S. Hong, C. Park, and J. Kim, ``Solar Power Forecasting Using CNN-LSTM Hybrid Model,'' \textit{Energies}, vol. 15, no. 21, p. 8233, 2022.

\bibitem{unlu2025comparative}
A. Unlu and M. Pe{\~n}a, ``Comparative Analysis of Hybrid Deep Learning Models for Electricity Load Forecasting During Extreme Weather,'' \textit{Energies}, vol. 18, no. 12, p. 3068, 2025.

\bibitem{sarmas2023short}
E. Sarmas, E. Spiliotis, E. Stamatopoulos, V. Marinakis, and H. Doukas, ``Short-term photovoltaic power forecasting using meta-learning and numerical weather prediction independent Long Short-Term Memory models,'' \textit{Renewable Energy}, 2023.

\bibitem{kumari2021deep}
P. Kumari and D. Toshniwal, ``Deep learning models for solar irradiance forecasting: A comprehensive review,'' \textit{Journal of Cleaner Production}, vol. 318, p. 128566, 2021.

\bibitem{ahmed2020review}
R. Ahmed, V. Sreeram, Y. Mishra, and M. Arif, ``A review and evaluation of the state-of-the-art in PV solar power forecasting: Techniques and optimization,'' \textit{Renewable \& Sustainable Energy Reviews}, vol. 124, p. 109792, 2020.

\bibitem{rajagukguk2020review}
R. Rajagukguk, R. Ramadhan, and H. Lee, ``A Review on Deep Learning Models for Forecasting Time Series Data of Solar Irradiance and Photovoltaic Power,'' \textit{Energies}, vol. 13, no. 24, p. 6623, 2020.

\bibitem{assaf2023review}
A. Assaf, H. Haron, H. Hamed, F. Ghaleb, S. Qasem, and A. Albarrak, ``A Review on Neural Network Based Models for Short Term Solar Irradiance Forecasting,'' \textit{Applied Sciences}, vol. 13, no. 14, p. 8332, 2023.

\bibitem{gupta2024review}
A. Gupta and R. Singh, ``A review of the state of the art in solar photovoltaic output power forecasting using data-driven models,'' \textit{Electrical Engineering}, 2024.

\bibitem{zhang2024review}
Q. Zhang, B. Lei, J. Zhou, and Y. Liu, ``A Review of Photovoltaic Power Generation Forecasting Techniques and Deep Learning Models,'' in \textit{2024 The 9th International Conference on Power and Renewable Energy (ICPRE)}, 2024, pp. 1342--1347.

\bibitem{gupta2021pv}
P. Gupta and R. Singh, ``PV power forecasting based on data-driven models: a review,'' \textit{International Journal of Sustainable Engineering}, vol. 14, no. 6, pp. 1733--1755, 2021.

\bibitem{pombo2022benchmarking}
D. Pombo, P. Bacher, C. Ziras, H. Bindner, S. Spataru, and P. Sorensen, ``Benchmarking physics-informed machine learning-based short term PV-power forecasting tools,'' \textit{Energy Reports}, 2022.

\bibitem{ferkous2024novel}
K. Ferkous, M. Guermoui, S. Menakh, A. Bellaour, and T. Boulmaz, ``A novel learning approach for short-term photovoltaic power forecasting - A review and case studies,'' \textit{Engineering Applications of Artificial Intelligence}, vol. 133, p. 108502, 2024.

\bibitem{lipu2021artificial}
M. Lipu, M. Miah, M. Hannan, A. Hussain, M. Sarker, A. Ayob, M. Saad, and M. Mahmud, ``Artificial Intelligence Based Hybrid Forecasting Approaches for Wind Power Generation: Progress, Challenges and Prospects,'' \textit{IEEE Access}, vol. 9, pp. 102460--102489, 2021.

\bibitem{dou2023comparison}
Y. Dou, S. Tan, and D. Xie, ``Comparison of machine learning and statistical methods in the field of renewable energy power generation forecasting: a mini review,'' \textit{Frontiers in Energy Research}, 2023.

\bibitem{hewage2020deep}
P. Hewage, M. Trovati, E. Pereira, and A. Behera, ``Deep learning-based effective fine-grained weather forecasting model,'' \textit{Pattern Analysis and Applications}, vol. 24, no. 1, pp. 343--366, 2020.

\bibitem{hachimi2024advancements}
C. Hachimi, S. Belaqziz, S. Khabba, B. Hssaine, M. Kharrou, and A. Chehbouni, ``Advancements in weather forecasting for precision agriculture: From statistical modeling to transformer-based architectures,'' \textit{Stochastic Environmental Research and Risk Assessment}, 2024.

\bibitem{xu2020data}
L. Xu, N. Chen, X. Zhang, and Z. Chen, ``A data-driven multi-model ensemble for deterministic and probabilistic precipitation forecasting at seasonal scale,'' \textit{Climate Dynamics}, vol. 54, no. 7, pp. 3355--3374, 2020.

\bibitem{schultz2021can}
M. Schultz, C. Betancourt, B. Gong, F. Kleinert, M. Langguth, L. Leufen, A. Mozaffari, and S. Stadtler, ``Can deep learning beat numerical weather prediction?'' \textit{Philosophical Transactions of the Royal Society A}, vol. 379, no. 2194, 2021.

\bibitem{vennila2022forecasting}
C. Vennila, A. Titus, T. Sudha, U. Sreenivasulu, N. Pandu, R. Reddy, K. Jamal, D. Lakshmaiah, P. Jagadeesh, and A. Belay, ``Forecasting Solar Energy Production Using Machine Learning,'' \textit{International Journal of Photoenergy}, 2022.

\bibitem{shahhosseini2020forecasting}
M. Shahhosseini, G. Hu, and S. Archontoulis, ``Forecasting Corn Yield With Machine Learning Ensembles,'' \textit{Frontiers in Plant Science}, vol. 11, p. 1120, 2020.

\bibitem{fatima2024review}
S. Fatima and A. Rahimi, ``A Review of Time-Series Forecasting Algorithms for Industrial Manufacturing Systems,'' \textit{Machines}, vol. 12, no. 6, p. 380, 2024.

\bibitem{makridakis2018statistical}
S. Makridakis, E. Spiliotis, and V. Assimakopoulos, ``Statistical and Machine Learning forecasting methods: Concerns and ways forward,'' \textit{PLoS ONE}, vol. 13, no. 3, p. e0194889, 2018.

\bibitem{szostek2024analysis}
K. Szostek, D. Mazur, G. Dra{\l}us, and J. Kusznier, ``Analysis of the Effectiveness of ARIMA, SARIMA, and SVR Models in Time Series Forecasting: A Case Study of Wind Farm Energy Production,'' \textit{Energies}, vol. 17, no. 19, p. 4803, 2024.

\bibitem{gu2017lightgbm}
G. Ke, Q. Meng, T. Finley, T. Wang, W. Chen, W. Ma, Q. Ye, and T.-Y. Liu, ``LightGBM: A Highly Efficient Gradient Boosting Decision Tree,'' in \textit{Advances in Neural Information Processing Systems}, vol. 30, 2017.

\bibitem{chen2016xgboost}
T. Chen and C. Guestrin, ``XGBoost: A Scalable Tree Boosting System,'' in \textit{Proceedings of the 22nd ACM SIGKDD International Conference on Knowledge Discovery and Data Mining}, 2016, pp. 785--794.

\bibitem{prokhorenkova2018catboost}
L. Prokhorenkova, G. Gusev, A. Vorobev, A. V. Dorogush, and A. Gulin, ``CatBoost: unbiased boosting with categorical features,'' in \textit{Advances in Neural Information Processing Systems}, vol. 31, 2018.

\bibitem{koenker2001quantile}
R. Koenker and K. F. Hallock, ``Quantile regression,'' \textit{Journal of Economic Perspectives}, vol. 15, no. 4, pp. 143--156, 2001.

\bibitem{akiba2019optuna}
T. Akiba, S. Sano, T. Yanase, T. Ohta, and M. Koyama, ``Optuna: A Next-generation Hyperparameter Optimization Framework,'' in \textit{Proceedings of the 25th ACM SIGKDD International Conference on Knowledge Discovery \& Data Mining}, 2019, pp. 2623--2631.

\bibitem{bergmeir2012note}
C. Bergmeir and J. M. Ben{\'i}tez, ``On the use of cross-validation for time series predictor evaluation,'' \textit{Information Sciences}, vol. 191, pp. 192--213, 2012.

\bibitem{rousseeuw1987silhouettes}
P. J. Rousseeuw, ``Silhouettes: a graphical aid to the interpretation and validation of cluster analysis,'' \textit{Journal of Computational and Applied Mathematics}, vol. 20, pp. 53--65, 1987.

\bibitem{rousseeuw1993alternatives}
P. J. Rousseeuw and A. M. Leroy, \textit{Robust Regression and Outlier Detection}. John Wiley \& Sons, 1993.

\bibitem{dormann2013collinearity}
C. F. Dormann, J. Elith, S. Bacher, C. Buchmann, G. Carl, G. Carr{\'e}, J. R. G. Marquez, B. Gruber, B. Lafourcade, P. J. Leit{\~a}o, T. M{\"u}nkem{\"u}ller, C. McClean, P. E. Osborne, B. Reineking, B. Schr{\"o}der, A. K. Skidmore, D. Zurell, and S. Lautenbach, ``Collinearity: a review of methods to deal with it and a simulation study evaluating their performance,'' \textit{Ecography}, vol. 36, no. 1, pp. 27--46, 2013.

\bibitem{dee2011era}
D. P. Dee, S. M. Uppala, A. J. Simmons, P. Berrisford, P. Poli, S. Kobayashi, U. Andrae, M. A. Balmaseda, G. Balsamo, P. Bauer, P. Bechtold, A. C. M. Beljaars, L. van de Berg, J. Bidlot, N. Bormann, C. Delsol, R. Dragani, M. Fuentes, A. J. Geer, L. Haimberger, S. B. Healy, H. Hersbach, E. V. H{\'o}lm, L. Isaksen, P. K{\aa}llberg, M. K{\"o}hler, M. Matricardi, A. P. McNally, B. M. Monge-Sanz, J.-J. Morcrette, B.-K. Park, C. Peubey, P. de Rosnay, C. Tavolato, J.-N. Th{\'e}paut, and F. Vitart, ``The ERA-Interim reanalysis: configuration and performance of the data assimilation system,'' \textit{Quarterly Journal of the Royal Meteorological Society}, vol. 137, no. 656, pp. 553--597, 2011.

\bibitem{houze2014cloud}
R. A. Houze, Jr., \textit{Cloud Dynamics}, 2nd ed. Academic Press, 2014.

\bibitem{grotjahn2016extreme}
R. Grotjahn, R. Black, R. Leung, M. F. Wehner, M. Barlow, M. Bosilovich, A. Gershunov, W. J. Gutowski, Jr., J. R. Gyakum, R. W. Katz, Y.-K. Lee, Y.-K. Lim, and Prabhat, ``North American extreme temperature events and related large scale meteorological patterns: a review of statistical methods, dynamics, modeling, and trends,'' \textit{Climate Dynamics}, vol. 46, no. 3-4, pp. 1151--1184, 2016.

\bibitem{wmo2018guide}
World Meteorological Organization, \textit{Guide to Meteorological Instruments and Methods of Observation}, WMO-No. 8, 2018.

\bibitem{tukey1977exploratory}
J. W. Tukey, \textit{Exploratory Data Analysis}. Addison-Wesley, 1977.
\end{thebibliography}
\addcontentsline{toc}{chapter}{References}
